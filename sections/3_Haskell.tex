\documentclass[../thesis.tex]{subfiles}
\graphicspath{{\subfix{../images/}}}
\doublespacing
\begin{document}

\chapter{Haskell}

We will now continue to discuss Haskell from the point of view of category theory.
We will show that the constructions in Haskell make sense from a categorical point of view, and how we can use category theory to explain why side-effects are allowed in a pure language like Haskell.

\section{The Hask category}

\section{Initial and terminal objects}

\section{Functors}

\section{Monads}

\section{Side-effects}

\end{document}
