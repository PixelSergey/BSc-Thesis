\documentclass[../thesis.tex]{subfiles}
\graphicspath{{\subfix{../images/}}}
\doublespacing
\begin{document}

\chapter{Introduction}

\section{Motivation}

Category theory is a branch of mathematics that aims to provide an abstract framework for dealing with function-like objects \cite{bct}.
We will soon define a \textit{category} as some mathematical structure that has objects, arrows between objects, with some notion of composition and a few reasonable assumptions.
At first, this definition will seem like it is too broad to reasonably describe anything.
After all, the definition will not require the ``arrows'' to be anything specific, such as set-theoretical functions or even functions at all.
Indeed, the goal of category theory is to work on a higher level of abstraction, ignoring many specific details of mathematical structures and instead proving more general facts that emerge from their properties. This is best explained with an example.

\newpage
\begin{ex}
\label{ex:three_statements}
Consider the following statements from three distinct branches of mathematics:


\begin{table}[h]
    \centering
    \begin{tabular}{|p{5cm}|p{5cm}|p{5cm}|}
        \hline
        Group theory  & Topology & Linear Algebra \\
        \hline
         Let $(G,*)$ be a group. Then $\forall g \in G$, $\exists! g^{-1}$ s.t. $g^{-1} * g = g * g^{-1}= e$.&
         Let $(X,\tau)$ and $(Y,\tau')$ be topological spaces. Let $f \colon X\to Y$ be a mapping. Then if $f^{-1}$ exists, it is unique.&
         Let $A\in \R^{n\times n}$ be a nonsingular matrix. If $B,C$ are both inverses of A, then $B=C$.\\
        \hline
    \end{tabular}
    \caption{Three statements about unique inverses}
    \label{tab:three_statements}
\end{table}

Each of these statements talks about unique inverses in its own specific case.
One might imagine that the proofs of each of these statements would look very different, as groups, toplogical spaces, and matrices are defined very differently.
However, let us look at each of the proofs side-by-side:

\begin{table}[h]
    \centering
    \begin{tabular}{|p{5cm}|p{5cm}|p{5cm}|}
        \hline
        Group theory  & Topology & Linear Algebra \\
        \hline
         Suppose $g$ has inverses $a,b$. Then $g*a=a*g=e$ and $g*b=b*g=e$. Thus $a=a*e=a*(g*b)=(a*g)*b=e*b=b$ \qed&
         Suppose $f$ has inverses $f_1^{-1},f_2^{-1}$. Then $f\circ f_1^{-1}=f_2^{-1}\circ f=\operatorname{Id}_X$ and $f\circ f_2^{-1}=f_1^{-1}\circ f=\operatorname{Id}_Y$. Thus $f_1^{-1}=f_1^{-1}\circ \operatorname{Id}_Y=f_1^{-1}\circ (f\circ f_2^{-1})=(f_1^{-1}\circ f)\circ f_2^{-1}=\operatorname{Id}_X \circ f_2^{-1}=f_2^{-1}$ \qed&
         Suppose $A$ has inverses $B,C$. Then $AB=BA=I_n$ and $AC=CA=I_n$. Thus $B=BI_n=B(AC)=(BA)C=I_nC=C$ \qed\\
        \hline
    \end{tabular}
    \caption{Proofs for the statements in Table \ref{tab:three_statements}}
    \label{tab:three_proofs}
\end{table}

One notices that the proofs, despite talking about distinct mathematical objects, all follow exactly the same structure.
The key point to notice in this example is that all three of these examples rely on the same two properties that we have proven in basic courses.
Firstly, the operations (binary operation of a group, composition of functions, and matrix multiplication) are all associative operations.
Secondly, all of these operations have an identity element (or two distinct left and right identities, in the case of the topological spaces). 
Associativity is used when we exchange the position of braces in the proofs, while the identity is used to introduce and then remove each of the inverse elements.

This example gives an indication that we are working on the wrong level of abstraction.
The proof of the statement ``inverses are unique'' happens on a higher level, where the details of each mathematical structure are not so important, while emergent properties, such as associativity and identities, are critical.
One can note that the proofs for associativity and existence of an identity will vary wildly between the three presented branches of mathematics.
This is because these proofs happen on a low level of abstraction, where details and definitions of mathematical objects are still important.
The point of category theory is to explore what facts we can prove on this higher level of abstraction.
Indeed, we will soon see that groups, topological spaces, and vector spaces are all examples of categories, and that the statement ``inverses are unique'' is, in fact, a category-theoretical result.
\qed
\end{ex}

One interesting application of category theory is in the Haskell programming language.
There are many concepts in Haskell named after constructions in category theory, such as Monads, Monoids, and Functors \cite{haskmooc}.
The goal of this thesis is to explain why these names actually carry category-theoretical meaning, and are not just invented out of nowhere.
We will see that it is possible to define a formal category \textbf{Hask} of Haskell's types and functions, and that all of the above constructions have a similarly-named equivalent in category theory.

\section{Basic definitions}

\begin{defn}
A \textbf{category} $\C$ consists of:
\begin{itemize}
    \item A class $\ob(\C)$ of \textbf{objects}
    \item For all $A,B\in\ob(\C)$, a class $\C(A,B)$ of \textbf{arrows} or \textbf{morphisms}
    \item For all pairs of morphisms $f\in \C(A,B), g\in \C(B,C)$,\\a \textbf{composition} $g \circ f \in \C(A,C)$
    \item For all objects $A\in \ob(\C)$, an \textbf{identity morphism} $1_A \in \C(A,A)$
\end{itemize}
Such that the following conditions are satisfied:
\begin{itemize}
    \item \textbf{Associativity:} for all $f\in \C(A,B),g\in \C(B,C),h\in \C(C,D)$,\\$h\circ (g\circ f)=(h\circ g)\circ f$
    \item \textbf{Identity:} for all $f\in \C(A,B)$, $f\circ 1_A=1_B\circ f=f$
\end{itemize}

\qed
\end{defn}

Note that in the above definition, we refer to the concept of a \textit{class}.
We use this to refer to any collection of mathematical objects, similar to a \textit{set}.
However, sets are too restrictive for the purposes of category theory.
For example, we will later want to define a category where the objects are all the sets, and the morphisms are set-theoretical functions.
However, it is known that the set of all sets cannot exist, as it contradicts the axioms of Zermelo-Fraenkel set theory (ZF).
On the contrary, it is possible to define traditionally contradictory notions like the class of all sets or even the class of all classes.
We will not formally define what a class is in this text, as the definition varies widely between different axiomatic frameworks.
In fact, a class is not even a well-defined object in many cases, such as in ZF.
For the purposes of this thesis we will ignore these details and consider a class to be simply some mathematical object, to which any other mathematical object can belong.

We will now give a few examples of categories that motivate the study of category theory.
We will see that categories are a very generic concept, and thus the theory that we will develop later has a very fundamental and wide range of applications.
The first three examples will formalise the intuition we developed in Example \ref{ex:three_statements}.

\begin{ex}
\label{ex:monoid}
\textit{A monoid is a category.}

Recall that a monoid is a group without inverses.
We will prove that for all monoids, we can construct a category with the same properties.
We will be able to extend this example to prove that any group is a category later, once we have developed enough theory about categories and inverse elements.

Formally, the pair $(G,*)$ is a monoid iff the following conditions hold:
\begin{itemize}
    \item Closure: $\forall a,b\in G$, $a*b\in G$
    \item Associativity: $\forall a,b,c \in G$, $(a*b)*c=a*(b*c)$
    \item Identity element: $\exists e\in G$ s.t. $\forall a\in G$, $e*a=a*e=a$
\end{itemize}

Let us define a new category $\M$ as follows:
\begin{itemize}
    \item Let $\ob(\M) = \{A\}$
    \item Let $\M(A,A) = G$
    \item Let $1_A=e$
    \item Define composition as follows: $\forall f,g\in \M(A,A)$, $g\circ f = f*g$
\end{itemize}

In other words, the class of objects of our category is just a singleton containing a dummy element.
We also take the class of morphisms to simply be the set $G$.
We must now make sure that our definition makes sense and satisfies the two axioms of being a category.
These follow immediately from the definition of a monoid.
Identity holds as $\forall f\in \M(A,A)$, $1_A \circ f = f*e = f = e*f = f \circ 1_A$.
Associativity holds as $\forall f,g,h\in \M(A,A)$, $h \circ (g \circ f) = (f*g)*h = f*(g*h) = (h \circ g) \circ f$.
Hence we have found a representation of our monoid as a category.
\qed
\end{ex}

Note that in the above example, we have done a common category-theoretical trick that I will refer to as ``going up one level''.
Instead of mapping the elements of our monoid to the objects of our category, we instead created a dummy object and mapped the elements of our monoid to the \textit{morphisms} of our category.
Then, the binary operation was converted into a \textit{composition of morphisms} instead of remaining as a morphism, or function, itself.
What this does is effectively ``moves us up'' by a level of abstraction: when discussing results in category theory, we generally want to disregard the specifics of what the objects themselves are, and instead focus on just how the morphisms compose with each other.

\begin{ex}
\label{ex:topological_space}
\textit{A topological space is a category.}
\qed
\end{ex}

\begin{ex}
\label{ex:vector_space}
\textit{A vector space is a category.}
\qed
\end{ex}

This gives us an indication as to why Example \ref{ex:three_statements} makes sense.
We will formally prove the statement in a theorem later.

Here is one final, quite important example of a category.

\begin{ex}
\label{ex:set}
\textit{Sets and functions form a category.}
\qed
\end{ex}



\end{document}
